\documentclass[UTF8]{ctexart}

\usepackage{geometry}
\usepackage{listings}
\usepackage{natbib}
\usepackage{graphicx}
\usepackage{subcaption}
\usepackage{algorithm}
\usepackage{amsmath}
\usepackage[noend]{algpseudocode}
\usepackage{lmodern}
\usepackage{url}
\usepackage{verbatim}
\usepackage{multirow}
\usepackage{hhline}

\lstset{frame=tb,
  aboveskip=3mm,
  belowskip=3mm,
  showstringspaces=false,
  columns=flexible,
  basicstyle={\small\ttfamily},
  numbers=none,
  numberstyle=\tiny\color{gray},
  breaklines=true,
  breakatwhitespace=true,
  tabsize=3
}

\geometry{left=4cm,right=4cm,top=3cm,bottom=3cm}

\title{单链表回路问题的三种算法的性能比较与分析}
\author{何昊}

\begin{document}

\maketitle

\section{引言}

链表是程序设计中最基本的数据结构之一。对于一个维护线性地址空间的程序而言,这个程序中的某个数据结构要么利用一段连续的内存,要么利用多段不连续的内存来保存其所需要的数据。对于后者而言,最简单的数据结构就是链表。链表虽然简单,但是在实际的算法和系统中有广泛的应用。例如,邻接链表可以用于表示一个图和解决哈希表的碰撞问题\cite{cormen2009introduction};链表可以用来实现动态内存管理\cite{bryant2003computer};等等。

对于以上使用链表的程序而言,保证链表的正确性很重要,其中尤其重要的是链表不能有环。在链表上运行的哪怕是最基本的算法,都会因为链表存在环而出现问题。例如,任何一个遍历链表的算法都会因为环而陷入死循环。因此,在程序运行时对链表进行高效的环路检查,对于测试软件正确性非常重要\cite{auguston1997assertions}。此外,链表回路的检测算法也可应用于伪随机数生成器分析\cite{knuth2014art}、密码学\cite{quisquater1989easy}、程序分析\cite{van1987efficient}、力学模拟\cite{fich1981lower}等等领域。

以上问题被称作\textbf{单链表回路问题}。已经存在多种高效算法可以用来解决单链表回路问题,其中比较著名的有\textbf{赛跑法}、\textbf{翻转法}和\textbf{指向自身法}三种。对这三种算法的性能的细致比较,将会有助于在实际应用场景中选择最合适的算法。因此,在本文中,我们对这三种算法的时间和空间性能,进行理论和实验上的比较与分析。我们发现,对于长度为$m+n$,环的长度为$n$的链表,三种算法的时间复杂度均为$O(m+n)$;赛跑法和翻转法的空间复杂度为$O(1)$,而指向自身法的空间复杂度为$O(m+n)$。在实验中,我们发现,在要求保留原链表结构和不允许内存泄露的前提下,赛跑法拥有最优秀的运行时间和运行时内存消耗;翻转法的内存消耗与赛跑法接近而运行时间略慢;指向自身法在运行时间上和内存消耗上都差于前两者。因此,本文建议在实际应用中,使用赛跑法来解决单链表回路问题。

本文的余下部分是这样组织的。第二节将会详细介绍三种解决单链表回路问题的算法;第三节将会对三种算法的理论性能,包括时间复杂度和空间复杂度进行分析;第四节将会从实验的角度评测这三种算法的实际时间和空间性能;最后,在第五节得出结论。

\section{解决方案介绍}

\subsection{赛跑法}

赛跑法的基本思想是,在遍历链表的过程中,使用一个快指针和慢指针从头开始遍历。在每次遍历时,快指针前进两个节点,慢指针前进一个节点。如果快指针到达链表结尾,则链表没有环。如果在某次迭代中快指针与慢指针指向同一个节点,则链表中有环。赛跑法不仅不需要额外内存空间,也不用对链表进行任何写操作。赛跑法的C++实现如下。

\begin{lstlisting}
struct Node {
    Node *next;
};
bool has_cycle_running(Node *lst) {
    Node *fast = lst;
    Node *slow = lst;
    while (true) {
        fast = fast->next;
        if (fast == nullptr) return false;
        if (fast == slow) return true;
        fast = fast->next;
        if (fast == nullptr) return false;
        if (fast == slow) return true;
        slow = slow->next;
        if (fast == slow) return true;
    }
}
\end{lstlisting}

\subsection{翻转法}

翻转法的基本思想是,在遍历链表的过程中,每次使用两个指针将链表的节点进行翻转,让子节点指向父节点。如果一直翻转到最后又回到了头部节点,则链表存在环。如果翻转到最后遇到一个没有子节点的节点,则链表不存在环。翻转法也不需要额外内存空间,且可以在运行完之后将链表恢复成原来的状态。翻转法的C++实现如下。

\begin{lstlisting}
bool reverse(Node *lst, Node **tail) {
    Node *prev = nullptr;
    Node *curr = lst;
    while (curr != nullptr) {
        Node *temp = curr->next;
        curr->next = prev;
        *tail = prev = curr;
        curr = temp;
        if (curr == lst) {
            curr->next = prev;
            return true;
        }
    }
    return false;
}

bool has_cycle_reverse(Node *lst) {
    Node *tail = nullptr;
    bool result = reverse(lst, &tail);
    if (result) reverse(lst, &tail);
    else reverse(tail, &tail);
    return result;
}
\end{lstlisting}

\subsection{指向自身法}

指向自身法的核心思想是,在遍历链表的过程中,将每个节点指向自身。如果在遍历途中遇到一个已经指向自身的节点,则链表存在环。如果没有遇到这样的节点而达到了链表结尾,则链表不存在环。这个算法运行完后,必须借助额外数据结构,才能将链表恢复成原来的状态,或者对链表节点进行垃圾回收。

\begin{lstlisting}
bool has_cycle_pointself(Node* lst) {
    bool result = false;
    Node *curr = lst;
    std::vector<Node *> visited;
    while (curr != nullptr) {
        if (curr->next == curr) {
            result = true;
            break;
        }
        visited.push_back(curr);
        Node *temp = curr->next;
        curr->next = curr;
        curr = temp;
    }
    for (Node *n : visited) delete n;
    return result;
}
\end{lstlisting}

\section{理论性能分析}

在本节中,我们利用大$O$表示法对三种算法的时间复杂度和空间复杂度进行分析。在单链表回路问题中,输入数据中任何一个链表都可以用唯一的整数二元组$\langle m, n \rangle$表示,其中$m$为链表中不是环的部分的长度,$n$为环的长度。对于没有环的链表而言,$n=0$。

对于赛跑法而言,我们分以下两种情况讨论。对于没有环的链表($n=0$),当快指针达到链表末尾时停止,时间复杂度为$O(m)$;对于有环的链表,当慢指针走过$m$个节点进入环时,快指针已经在环里,此时快指针最多走过$n$个节点就能追上慢指针,因此慢指针最多走过$m+n/2$个节点后算法停止,时间复杂度为$O(m+n)$。赛跑法需要的额外的空间只有两个指针和若干临时变量,因此空间复杂度为$O(1)$。

对于翻转法而言,我们也分成两种情况讨论。对于没有环的链表($n=0$),算法在走到链表末尾时停止,时间复杂度为$O(m)$;对于有环的链表,算法会回到链表开头时停止,遍历$2m+n$个节点,时间复杂度为$O(m+n)$。翻转法需要的额外空间也只有两个指针和若干临时变量,因此空间复杂度为$O(1)$。

对于指向自身法而言,对于没有环的链表和有环的链表,算法都会在遍历$m+n$个节点后停止,因此时间复杂度为$O(m+n)$。然而,如果需要恢复链表或对链表进行存储空间回收,算法在运行过程中必须保存遍历过的节点,因此空间复杂度为$O(m+n)$。

三种算法的时间复杂度和空间复杂度总结见表\ref{tab:complexity}。

\begin{table}
\centering
\begin{tabular}{lll}
 算法 & 时间复杂度 & 空间复杂度 \\ 
\hhline{===}
赛跑法 & $O(m+n)$ & $O(1)$  \\ 
\hline
翻转法 & $O(m+n)$ & $O(1)$  \\ 
\hline
指向自身法 & $O(m+n)$ & $O(m+n)$
\end{tabular}
\caption{三种单链表回路算法的时间复杂度与空间复杂度}
\label{tab:complexity}
\end{table}

\section{实际性能评测}

\subsection{实验设计}

\subsection{实验过程}

\subsection{实验结果}

\section{结论}


\bibliographystyle{plain}
\bibliography{references}
\end{document}